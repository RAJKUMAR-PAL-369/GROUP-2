\documentclass[a4paper,12pt]{article}
\usepackage[utf8]{inputenc}
\usepackage{graphicx}
\usepackage{hyperref}
\usepackage{amsmath}
\usepackage{fancyhdr}

% Page formatting
\pagestyle{fancy}
\fancyhf{}
\fancyhead[L]{Software Tools And Technology}
\fancyfoot[C]{\thepage}

\begin{document}

% Title Page
\begin{center}
    \Large \textbf{Software Tools And Technology} \\[1cm]
\begin{figure}[h!]
   \centering
    \includegraphics[width=0.3\linewidth]{makaut logo.png}
\end{figure}
\vspace{0.5 cm}
    
    Group 2: \\[1cm]
    \textbf{Lab Notebook} \\[2cm]

    \vspace{0.1 cm}

    
    \textbf{Group members:} \\[0.5cm]
    1. Rajkumar Pal - Leader (BCA) \\[0.2cm]
    2. Jahir Mondal BSc IT(AI) \\[0.2cm]
    3. Arnab Pandit BSc IT(CS) \\[0.2cm]
    4. Pritam Pratihar (BCA) \\[0.2cm]
    5. Suhana Pervaze (BCA) \\[2cm]

    \textbf{Instructor}: Dr. Ayan Ghosh \\[0.5cm]
    Course: Software Tools And Technology
\end{center}

% Lab Notebook Entries
\section*{Lab Notebook Entries Index}

\subsection*{1. Lab Entry by Rajkumar Pal}
\subsubsection*{1.1 Experiment}
\begin{tabbing}
    Sl. No. \hspace{2cm} \= Assignments \\
    1. \> Introduction to Github and Github desktop version installation \\
\end{tabbing}

\subsection*{2. Lab Entry by Jahir Mondal}
\subsubsection*{2.1 Experiment}
\begin{tabbing}
    Sl. No. \hspace{2cm} \= Assignments \\
    1. \> Converting Submit button to Chin Tapak Dum Dum \\
\end{tabbing}

\subsection*{3. Lab Entry by Arnab Pandit}
\subsubsection*{3.1 Experiment}
\begin{tabbing}
    Sl. No. \hspace{2cm} \= Assignments \\
    1. \> Making calculator in C \\
\end{tabbing}

\subsection*{3. Lab Entry by Pritam Pratihar}
\subsubsection*{3.1 Experiment}
\begin{tabbing}
    Sl. No. \hspace{2cm} \= Assignments \\
    1. \> Creating LaTeX Repository in GitHub \\
\end{tabbing}

\subsection*{3. Lab Entry by Suhana Pervaze}
\subsubsection*{3.1 Experiment}
\begin{tabbing}
    Sl. No. \hspace{2cm} \= Assignments \\
    1. \> Introduction to LaTeX \\
\end{tabbing}

\newpage

% Entry 1: Rajkumar Pal
\section*{Lab Entry By Rajkumar Pal}

\subsection*{1. Introduction to GitHub and GitHub Desktop Version Installation}


In this section, we will explore what GitHub is, its features, and how to install the GitHub Desktop version. 
GitHub is a platform that enables developers to collaborate, track, and manage code versions through Git version control. 
It provides a web-based interface for easier management of Git repositories. 
\vspace{0.1 cm}
\begin{figure}[h!]
   \centering
    \includegraphics[width=0.5\linewidth]{25231.png}
\end{figure}
\vspace{0.5 cm}

\subsection*{1.1 Introduction to GitHub}
GitHub is a web-based platform that allows developers to host and review code, manage projects, and build software collaboratively. 
Git is a version control system that enables tracking of changes, and GitHub adds a graphical interface and additional features on top of Git.

\subsection*{1.2 GitHub Desktop Installation}
GitHub Desktop is a GUI client that simplifies the interaction with Git and GitHub. 
It allows users to handle repositories, create branches, and manage commits without using the command line.
\\
\\


\noindent \textbf{Steps to Install GitHub Desktop:}
\begin{enumerate}
    \item Visit the GitHub Desktop official website at \url{https://desktop.github.com/}.
    \item Click on the "Download for your OS" button (Windows/Mac).
    \item Follow the installation instructions provided by the installer.
    \item Once installed, sign in with your GitHub credentials to start using GitHub Desktop.
\end{enumerate}

\newpage
% Entry 2: Jahir Mondal
\section*{Lab Entry By Jahir Mondal}
\subsection*{2. Changing the Submit Button to Chin Tapak Dum Dum and fixing the disproportionate}
Changed the submit button to "Chin Tapak Dum Dum" through this code.

\begin{figure}[h!]
    \centering
    \includegraphics[width=0.8\textwidth]{code.jpeg} % Replace with actual image
    \caption{JAVA CODE}
\end{figure}

\noindent\textbf{Modifications:}
\begin{itemize}
    \item Font Size and Style: The font size and style have been adjusted for improved readability and consistency with the overall design.
    \item Background Color: The background color of the button has been updated to create a more visually appealing and cohesive look.
    \item Font Color: The font color has been modified to ensure strong contrast with the background, enhancing legibility.
    \item Element Proportions: Any disproportionate elements have been corrected to achieve a more balanced and aesthetically pleasing design.
\end{itemize}

\begin{figure}[h!]
    \centering
    \includegraphics[width=0.8\textwidth]{output.jpeg} % Replace with actual image
    \caption{FINAL OUTPUT}
\end{figure}

\vspace{0.5cm}
\newpage

\end{document}
